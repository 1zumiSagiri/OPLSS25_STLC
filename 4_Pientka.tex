\documentclass[11pt]{article}
\usepackage{graphicx}
\usepackage{amsmath}
\usepackage{amssymb}
\usepackage[top=0.75in, bottom=1.25in, left=1in, right=1in,includefoot]
{geometry}
\usepackage{fancyhdr}
\usepackage{mathpartir}
\usepackage{appendix}
\usepackage{wasysym}
\usepackage{stmaryrd}
\usepackage[backend=biber,style=numeric,sorting=none]{biblatex} % bibliography, install biber otherwise it will not work
\addbibresource{4_Pientka.bib}
\usepackage[colorlinks=true, linkcolor=cyan, citecolor=cyan, urlcolor=blue]{hyperref}
\usepackage{blindtext}
\usepackage{bbm}
\newcommand\notate[1]{\quad\rlap{$#1$}}
%remove the line above

\DefineBibliographyStrings{english}{
    % add bibliography page to table of contents
    bibliography = {References},
}

\newcommand{\valid}[1]{\ensuremath{#1 :\textsf{valid}}}
\newcommand{\true}[1]{\ensuremath{#1 \textsf{true}}}
\newcommand{\projl}[1]{\ensuremath{\textsf{proj}_1\mkern2mu#1}}
\newcommand{\abs}[2]{\ensuremath{\lambda #1.\,#2}}
\newcommand{\letin}[2]{\ensuremath{\textsf{let}~#1 \mathrel{\textsf{in}} #2}}
\newcommand{\BOX}[1]{\ensuremath{\textsf{box}~#1}}
\newcommand{\clo}[1]{\ensuremath{\textsf{clo}~#1}}
\newcommand{\hole}{\fbox{\rule{1.5em}{0pt}\rule{0pt}{0.6em}}\ }

\pagestyle{fancy}
\fancyhead[L]{\fancyhdrbox[c]{\Large \slshape \sffamily 
Contextual Modal Type Theory}}
\fancyhead[C]{}
\fancyhead[R]{\large \sffamily \bfseries Brigitte Pientka}

\fancyfoot[L]{\fancyhdrbox[b]{\sffamily \small Compiled By: \\ 
\normalsize \sffamily Adam Brohl, Ellen Whalen, Vincent Chan}}
%Change the names above to your names

\fancyfoot[C]{\thepage}
\fancyfoot[R]{\fancyhdrbox[b]{\includegraphics[height=0.35in]{oplssLogo.png}}}
\renewcommand{\footrulewidth}{0.5pt}

\setlength{\parindent}{0pt}
\setlength{\parskip}{2ex}
\setlength{\headheight}{1.25in}

\begin{document}
\thispagestyle{plain}
\begin{center}
\includegraphics[width=3in]{oplssLogo.png}\\[2\parskip]
\sffamily \LARGE \slshape Contextual Modal Type Theory
--- \upshape Brigitte Pientka \\[2ex]
\large Lecture 4 - \slshape June 26, 2025
\end{center}

\section{Notations}
As discussed in the previous lecture, the following notations are used:
\begin{itemize}
    \item $\Box A : A$ is necessarily true / $A$ is valid
    \item $\Delta$ : context of `global' valid assumptions, `live forever', $\{\valid{A_1},\ldots,\valid{A_n}\}$
    \item $\Gamma$ : context of `local' true assumptions, `live here and now', $\{\true{A_1 :},\ldots,\true{A_n :}\}$
\end{itemize}

\section{Contextual Types}
\subsection{Intro to contextual types}
\begin{itemize}
    \item We'll examine a common example from natural deduction: \[x : A \supset B \supset C, y : A \land B \vdash \hole : B \supset C\]
    \item Usually, we would like to fill the hole by applying \projl{B} to $A$ to get $B \supset C$
    \item However, \hole \, here is not necessarily closed
    \item The idea of contextual type is to pair the context (here is $x : A \supset B \supset C, y : A \land B$) with the conclusion (here is $B \supset C$)\cite{Nanevski2008ContextualModalTypeTheory}
    \item Think about when we type check the following:
        \begin{equation}\label{ex1}
            \infer*{x : \textsf{int} \vdash \hole + 1: \textsf{int}}{\cdot \vdash \abs{x}{\hole + 1} : \textsf{int}}
        \end{equation}
    \item We know that $\hole$ stands for $\textsf{int}$ in the context of $x : \textsf{int}$
\end{itemize}

\noindent\textbf{Types/Props}
$A := \ldots | \, \Box(\Psi \Vdash A)$
\\We can read $\Psi \Vdash A$ as ``pair the context $\Psi$ with the conclusion $A$''. 

\noindent\textbf{Terms}
$M := \ldots | \, \BOX{(\Psi, M)}$

Some possible inhabitants of \hole given example \ref{ex1}:
\begin{itemize}
    \item 0
    \begin{itemize}
        \item A note: the term $\BOX(x:\textsf{int, 0})$ has type $\Box(x: \textsf{int} \Vdash \textsf{int})$
    \end{itemize}
    \item $x$ (because we already have that $x: \textsf{int}$)
    \item $x * x$
\end{itemize} 

Takeaway: we have a typing derivation that we haven't yet finished, represented by our \hole s. Thus we refer to the conclusion to infer the contextual type.
\subsection{Contextual rule rewrites} 
\begin{minipage}{3in}
    \begin{mathpar}
        \infer*{\Delta; \Psi \vdash M:A}{\Delta; \Gamma \vdash \BOX(\Psi, M): \Box(\Psi \Vdash A)} \notate{(\Box I)}
    \end{mathpar}
\end{minipage}
\hfill
\begin{minipage}{3in}
    \begin{mathpar}
        \infer*{\Delta; \Gamma \vdash M: \Box(\Psi \Vdash A) \; \; \; \; (\Delta, u:(\Psi \Vdash A)); \Gamma \vdash N}{\Delta; \Gamma \vdash \letin{\BOX u = M}{N:C}} \notate{(\Box E)}
    \end{mathpar}
\end{minipage}

\medskip

\begin{minipage}{3in}
    \begin{mathpar}
        \infer*{\true{x: A} \in \Gamma}{\Delta; \Gamma \vdash x: A}
    \end{mathpar}
\end{minipage}
\begin{minipage}{3in}
\begin{mathpar}
    \infer*{u: (\Psi \Vdash A) \in \Delta \; \; \; \; \Delta; \Gamma \vdash \sigma: \Psi}{\Delta; \Gamma \vdash \clo{(u, \sigma)}:A}
\end{mathpar}
\end{minipage}

We define the relation between $\Psi$ and $\Gamma$ by providing witnesses: 
\smallskip
\\in equation $\Delta; \Gamma \vdash \sigma: \Psi,$
\begin{itemize}
    \item $\sigma = M_1/x_1, \,\ldots, M_n/x_n$
    \item $\Psi = \true{x_1:A_1}, \,\ldots, \true{x_n: A_n}$
\end{itemize}
where $\Delta; \Gamma \vdash M_i: A_i$.

In more formal terms,
\begin{mathpar}
    \infer*{\Delta; \Gamma \vdash \sigma: \Psi \;\;\;\; \Delta; \Gamma \vdash M:A}{\Delta; \Gamma \vdash (\sigma, M/x): \Psi, x:A}
\end{mathpar}

In essence, $\Psi$ is the domain, $\Gamma$ is the range, and $\sigma$ is a mapping from $\Psi \longrightarrow \Gamma$. We provide instantiations for all $x \in \Psi,$ then make sure that all of these instantiations make sense in $\Gamma$.

Intuitively, doesn't it make sense that $\Box(x:A, y:B \Vdash C)$ is ``equivalent'' to $\Box(A\supset B\supset C)$? A small derivation:
\begin{mathpar}
    \infer*
    {\Delta; x: A, y: B \vdash \,\ldots \,C}
    {\infer*{\Delta; x: A \vdash \, \ldots \, B \supset C}
    {\infer*{\Delta; \cdot \vdash \, \ldots \, : A \supset B \supset C}
    {\Delta; \Gamma \vdash \, \ldots \, : \Box A \supset B \supset C}}}
\end{mathpar}

\section{Comparing contextual and non-contextual types}
\subsection{Example 1: implication chains}
\textbf{Type: } $\Box(C \supset A) \supset \Box(C \supset D \supset A)$
\smallskip
\\\textbf{Term:} $\lambda x: \Box(C \supset A). \letin{\BOX u = x}{\BOX(\lambda y: C. \lambda x: D. \; u \,\, y)}$
\; \; \; (where $u = C \supset A \textsf{ valid}$)

We can see that our $\BOX$ holds instructions followed by a function application: $u$ applied to $y$.

Now using contextual types: 
\smallskip \\ 
\textbf{Type: } $\Box(x': C \Vdash A) \supset \Box(y:C, x:D \Vdash A)$
\smallskip
\\ \textbf{Term: } 
\\ $\lambda x: \Box(x': C \Vdash A). \; \letin{\BOX u = x}{\BOX(y:C, x:D. \; \clo{(u, y/x')})}$ \; (where $u = (x': C \Vdash A))$

In this example, we turn $x'$s into $y$s using a closure (\textsf{clo}) instead of a function application. Closures fire much sooner than function applications, which we'll explore more in the following example.

\subsection{Example 2: \textsf{nth} function}
Rewriting our $\textsf{nth}$ function from the previous lecture, we have type signature 
$$\textsf{nth}: \textsf{int} \longrightarrow \Box(v: \textsf{bool\_vec} \Vdash \textsf{bool})$$ where
\begin{align*}
    \textsf{nth } 0 & = \BOX(v: \textsf{bool\_vec},\,\textsf{hd } v) \\
    \textsf{nth } (\textsf{suc } n) & = \letin{\BOX u = \textsf{nth }n}{\BOX(v: \textsf{bool\_vec}, \clo{(u, \textsf{tl }v)})}.
\end{align*}

Let's evaluate an example, $\textsf{nth } 2$:
\begin{align*}
    \textsf{nth }2 & \longrightarrow  \letin{\BOX u = \textsf{nth }1}{\BOX(v:{bool\_vec}, \clo{(u, \textsf{tl }v/v_1)})}\\
    \textsf{nth } 1 & \longrightarrow \letin{\BOX u = \textsf{nth }0}{\BOX(v:{bool\_vec}, \clo{(u, \textsf{tl }v/v_0)})}\\
    &\longrightarrow \letin{\BOX u = \BOX(v_0, \textsf{hd }v_0)}{\BOX(v:{bool\_vec}, \clo{(u, \textsf{tl }v/v_0)})}\\
    &\longrightarrow \BOX(v, \clo{(\textsf{hd }v_0, \textsf{tl }v/v_0)}) \\
    &\longrightarrow \BOX(v. \textsf{hd}(\textsf{tl }v))
\end{align*}

We can see that our contextual-typed version is eager, thus giving us the expected answer that our lazy function-evaluated \textsf{nth } function failed to show in the previous lecture. Functions evaluate once they have all of the information; contextual types treat arguments like syntax and splice until they get the most simplified result.
\begin{itemize}
    \item Substitution for ``local'' variables: $[M/x]x = M$
    \item Modal substitution for global variables: $\llbracket (\Psi. M)/u\rrbracket \clo{(u, \sigma)} = [\sigma] M$
\end{itemize}

One example of where this is used is in simplifying expressions like $\BOX{(\textsf{fn } x \rightarrow x + (x + 0))}$. We want our program to continue to analyze this expression, resulting in the simpler $\BOX{(\textsf{fn } x \rightarrow x + x)}$. However, the program can't \textit{evaluate} because we don't have a value of $x$ provided. Thus, we use syntax manipulation - $(x+ 0)$ is itself a contextual type.

\section{Another view: ``quote and splice''}
In the Kripke-like ``quote and splice'' view, we have expressions like 
\[\Gamma_1; \, \ldots \, ; \Gamma_n \vdash M:A\]
and rules like
\\
\begin{minipage}{3in}
    \begin{mathpar}
        \infer*{\overrightarrow{\Gamma}; \Gamma; \cdot \vdash M:A}{\overrightarrow{\Gamma}; \Gamma \vdash \Box{M: \Box A}}
    \end{mathpar}
\end{minipage}
\begin{minipage}{3in}
\begin{mathpar}
    \infer*{\overrightarrow{\Gamma}; \Gamma \vdash M: \Box A}{\overrightarrow{\Gamma}; \Gamma_i; \, \ldots\, ; \Gamma_n \vdash \textsf{unbox}_n M: A}
\end{mathpar}
\end{minipage}
The two views are proven to be equivalent. The Kripke view is more akin to code, but the contextually-typed view is more well-behaved.

\newpage
\printbibliography[heading=bibintoc]
\end{document}
