\documentclass[11pt]{article}
\usepackage{graphicx}
\usepackage{amsmath}
\usepackage{amssymb}
\usepackage[top=0.75in, bottom=1.25in, left=1in, right=1in,includefoot]
{geometry}
\usepackage{fancyhdr}
\usepackage{mathpartir}
\usepackage{appendix}
\usepackage[backend=biber,style=numeric,sorting=none]{biblatex} % bibliography, install biber otherwise it will not work
\addbibresource{4_Pientka.bib}
\usepackage[colorlinks=true, linkcolor=cyan, citecolor=cyan, urlcolor=blue]{hyperref}
\usepackage{blindtext}
\usepackage{bbm}
%remove the line above

\DefineBibliographyStrings{english}{
    % add bibliography page to table of contents
    bibliography = {References},
}

\newcommand{\valid}[1]{\ensuremath{#1 :\textsf{valid}}}
\newcommand{\true}[1]{\ensuremath{#1 :\textsf{true}}}
\newcommand{\projl}[1]{\ensuremath{\textsf{proj}_1\mkern2mu#1}}
\newcommand{\abs}[2]{\ensuremath{\lambda #1.\,#2}}
\newcommand{\letin}[2]{\ensuremath{\textsf{let}~#1 \mathrel{\textsf{in}} #2}}
\newcommand{\BOX}[1]{\ensuremath{\textsf{box}~#1}}
\newcommand{\clo}[1]{\ensuremath{\textsf{clo}~#1}}
\newcommand{\hole}{\fbox{\rule{1.5em}{0pt}\rule{0pt}{0.6em}}\ }

\pagestyle{fancy}
\fancyhead[L]{\fancyhdrbox[c]{\Large \slshape \sffamily 
Contextual Modal Type Theory}}
\fancyhead[C]{}
\fancyhead[R]{\large \sffamily \bfseries Brigitte Pientka}

\fancyfoot[L]{\fancyhdrbox[b]{\sffamily \small Compiled By: \\ 
\normalsize \sffamily Adam Brohl, Ellen Whalen, Vincent Chan}}
%Change the names above to your names

\fancyfoot[C]{\thepage}
\fancyfoot[R]{\fancyhdrbox[b]{\includegraphics[height=0.35in]{oplssLogo.png}}}
\renewcommand{\footrulewidth}{0.5pt}

\setlength{\parindent}{0pt}
\setlength{\parskip}{2ex}
\setlength{\headheight}{1.25in}

\begin{document}
\thispagestyle{plain}
\begin{center}
\includegraphics[width=3in]{oplssLogo.png}\\[2\parskip]
\sffamily \LARGE \slshape Lambda-Calculi for Logics
--- \upshape Valeria De Paiva \\[2ex]
\large Lecture 4 - \slshape June 26, 2025
\end{center}

\section{Notations}
As discussed in the previous lecture, the following notations are used:
\begin{itemize}
    \item $\Box A : A$ is necessarily true / $A$ is valid
    \item $\Delta$ : context of `global' valid assumptions, `live forever', $\{\valid{A_1},\ldots,\valid{A_n}\}$
    \item $\Gamma$ : context of `local' true assumptions, `live here and now', $\{\true{A_1},\ldots,\true{A_n}\}$
\end{itemize}

\section{Contextual Type}
\begin{itemize}
    \item We use a common example from natural deduction: \[x : A \supset B \supset C, y : A \land B \vdash \hole : B \supset C\]
    \item Usually, we would like to fill the hole by applying \projl{B} to $A$ to get $B \supset C$
    \item However, \hole here is not necessarily closed
    \item The idea of contextual type is to pair the context (here is $x : A \supset B \supset C, y : A \land B$) with the conclusion (here is $B \supset C$)
    \item Think about when we type check the following:
        \begin{mathpar}
            \infer*{x : \textsf{int} \vdash \hole + 1: \textsf{int}}{\cdot \vdash \abs{x}{\hole + 1} : \textsf{int}}
        \end{mathpar}
    \item We know that $\hole$ stands for $\textsf{int}$ in the context of $x : \textsf{int}$
\end{itemize}

\noindent\textbf{Types/Props}
$A := \ldots | \Box(\Psi \Vdash A)$

\noindent\textbf{Terms}
$M := \ldots | \BOX{(\Psi, M)}$


\newpage
\printbibliography[heading=bibintoc]
\end{document}
